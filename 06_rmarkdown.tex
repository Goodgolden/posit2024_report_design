% Options for packages loaded elsewhere
\PassOptionsToPackage{unicode}{hyperref}
\PassOptionsToPackage{hyphens}{url}
%
\documentclass[
  ignorenonframetext,
]{beamer}
\usepackage{pgfpages}
\setbeamertemplate{caption}[numbered]
\setbeamertemplate{caption label separator}{: }
\setbeamercolor{caption name}{fg=normal text.fg}
\beamertemplatenavigationsymbolsempty
% Prevent slide breaks in the middle of a paragraph
\widowpenalties 1 10000
\raggedbottom
\setbeamertemplate{part page}{
  \centering
  \begin{beamercolorbox}[sep=16pt,center]{part title}
    \usebeamerfont{part title}\insertpart\par
  \end{beamercolorbox}
}
\setbeamertemplate{section page}{
  \centering
  \begin{beamercolorbox}[sep=12pt,center]{part title}
    \usebeamerfont{section title}\insertsection\par
  \end{beamercolorbox}
}
\setbeamertemplate{subsection page}{
  \centering
  \begin{beamercolorbox}[sep=8pt,center]{part title}
    \usebeamerfont{subsection title}\insertsubsection\par
  \end{beamercolorbox}
}
\AtBeginPart{
  \frame{\partpage}
}
\AtBeginSection{
  \ifbibliography
  \else
    \frame{\sectionpage}
  \fi
}
\AtBeginSubsection{
  \frame{\subsectionpage}
}
\usepackage{amsmath,amssymb}
\usepackage{iftex}
\ifPDFTeX
  \usepackage[T1]{fontenc}
  \usepackage[utf8]{inputenc}
  \usepackage{textcomp} % provide euro and other symbols
\else % if luatex or xetex
  \usepackage{unicode-math} % this also loads fontspec
  \defaultfontfeatures{Scale=MatchLowercase}
  \defaultfontfeatures[\rmfamily]{Ligatures=TeX,Scale=1}
\fi
\usepackage{lmodern}
\usetheme[]{CUDenver}
\ifPDFTeX\else
  % xetex/luatex font selection
\fi
% Use upquote if available, for straight quotes in verbatim environments
\IfFileExists{upquote.sty}{\usepackage{upquote}}{}
\IfFileExists{microtype.sty}{% use microtype if available
  \usepackage[]{microtype}
  \UseMicrotypeSet[protrusion]{basicmath} % disable protrusion for tt fonts
}{}
\makeatletter
\@ifundefined{KOMAClassName}{% if non-KOMA class
  \IfFileExists{parskip.sty}{%
    \usepackage{parskip}
  }{% else
    \setlength{\parindent}{0pt}
    \setlength{\parskip}{6pt plus 2pt minus 1pt}}
}{% if KOMA class
  \KOMAoptions{parskip=half}}
\makeatother
\usepackage{xcolor}
\newif\ifbibliography
\setlength{\emergencystretch}{3em} % prevent overfull lines
\providecommand{\tightlist}{%
  \setlength{\itemsep}{0pt}\setlength{\parskip}{0pt}}
\setcounter{secnumdepth}{-\maxdimen} % remove section numbering
\ifLuaTeX
  \usepackage{selnolig}  % disable illegal ligatures
\fi
\usepackage{bookmark}
\IfFileExists{xurl.sty}{\usepackage{xurl}}{} % add URL line breaks if available
\urlstyle{same}
\hypersetup{
  pdftitle={Rmarkdown and coding skills},
  pdfauthor={Randy (Xin) Jin},
  hidelinks,
  pdfcreator={LaTeX via pandoc}}

\title{Rmarkdown and coding skills}
\author{Randy (Xin) Jin}
\date{2024-08-20}
\institute{Biostatistic and Bioinformatic Department}

\begin{document}
\frame{\titlepage}

\begin{frame}[allowframebreaks]
  \tableofcontents[hideallsubsections]
\end{frame}
\section{RMarkdown}\label{rmarkdown}

\begin{frame}{\href{http://rmarkdown.rstudio.com}{RMarkdown}}
\phantomsection\label{rmarkdown-1}
\begin{itemize}
\item
  R Markdown documents are fully reproducible.
\item
  Use a productive notebook interface to weave together narrative text
  and code to produce elegantly formatted output.
\item
  Use multiple languages including R, Python, and SQL.
\item
  Support output formats including HTML, PDF, MS Word, Beamer, books,
  dashboards, shiny applications, scientific articles, websites, and
  more.
\end{itemize}
\end{frame}

\begin{frame}{YAML}
\phantomsection\label{yaml}
There are three basic components of an R Markdown document: the
metadata, text, and code.

\begin{itemize}
\item
  The syntax for the metadata is YAML
\item
  Also called the YAML metadata or the YAML frontmatter
\item
  Do not forget to indent the sub-fields of a top field properly
\item
  Here is the YAML for this beamer slides
\end{itemize}
\end{frame}

\begin{frame}[fragile]{YAML}
\phantomsection\label{yaml-1}
\begin{verbatim}
---
title: "Rmarkdown and coding skills"
author: "Randy (Xin) Jin"
date: "2024-08-20"
institute: Biostatistic and Bioinformatice Department
output:
  beamer_presentation:
    toc: true
    keep_tex: yes
    slide_level: 2
    latex_engine: xelatex
    theme: CUDenver
---
\end{verbatim}
\end{frame}

\begin{frame}[fragile]{Text}
\phantomsection\label{text}
Inline LaTeX equations can be written in a pair of dollar signs using
the LaTeX syntax, e.g.,
\texttt{\$f(k)\ =\ \{n\ \textbackslash{}choose\ k\}\ p\^{}\{k\}\ (1-p)\^{}\{n-k\}\$}

And in real file, it look like this
\(f(k) = {n \choose k} p^{k} (1-p)^{n-k}\)

The math expressions of the display style can be written in a pair of
double dollar signs, e.g.,
\texttt{\$\$f(k)\ =\ \{n\ \textbackslash{}choose\ k\}\ p\^{}\{k\}\ (1-p)\^{}\{n-k\}\$\$},

And the output looks like this
\[f(k) = {n \choose k} p^{k} (1-p)^{n-k}\]
\end{frame}

\begin{frame}[fragile]{Text}
\phantomsection\label{text-1}
For example type the matrix or array

\begin{verbatim}
$$
\Theta = 
\begin{pmatrix}
  \alpha & \beta\\
  \gamma & \delta
\end{pmatrix}
$$
\end{verbatim}

\[\Theta = \begin{pmatrix}\alpha & \beta\\
\gamma & \delta
\end{pmatrix}\]
\end{frame}

\begin{frame}{Code}
\phantomsection\label{code}
The syntax for text (also known as prose or narratives) is Markdown

\begin{itemize}
\item
  A code chunk starts with three backticks
\item
  An inline R code expression
\item
  Remember to use the
  \href{https://www.rstudio.com/resources/cheatsheets/}{cheat sheet} and
  Help Reference
\end{itemize}
\end{frame}

\begin{frame}[fragile]{Render/Knitr}
\phantomsection\label{renderknitr}
The usual way to compile an R Markdown document is to click the
\textbf{Knit} button

The corresponding keyboard shortcut is \textbf{Ctrl + Shift + K}
(\textbf{Cmd + Shift + K} on macOS).

Under the hood, RStudio calls the function
\textbf{\texttt{rmarkdown::render()}} to render the document in a new R
session.
\end{frame}

\begin{frame}{Templates and shortcuts}
\phantomsection\label{templates-and-shortcuts}
Let's see the templates and shortcuts in Rstudio \ldots{}
\end{frame}

\section{RStudio Snippets}\label{rstudio-snippets}

\begin{frame}{Snippets}
\phantomsection\label{snippets}
Code snippets are text macros that are used for quickly inserting common
snippets of code

If you select the snippet from the completion list, it will be inserted
along with several text placeholders which you can fill in by typing and
then pressing Tab to advance to the next placeholder.

\begin{itemize}
\item
  By default, the completion list will show up automatically and
  activated via the \textbf{Tab} key.
\item
  If you have typed the character, then you can press
  \textbf{Shift+Tab}.
\end{itemize}

Note that for Markdown snippets within R Markdown documents too
\end{frame}

\begin{frame}[fragile]{Customizing Snippets}
\phantomsection\label{customizing-snippets}
You can edit the built-in snippet definitions and even add snippets of
your own via the \textbf{Edit Snippets} button in \textbf{Global Options
-\textgreater{} Code}

For example, try to type \texttt{matrix} and press \textbf{Shift+Tab} in
Latex

Or try to type \texttt{title} in the Rmarkdown and press
\textbf{Shift+Tab}
\end{frame}

\begin{frame}[fragile]{Saving and Sharing Snippets}
\phantomsection\label{saving-and-sharing-snippets}
Once you've customized snippets for a given language, they are written
into the \texttt{\textasciitilde{}/.R/snippets}directory.

For example, the customized versions of \texttt{R} and \texttt{C/C++}
snippets are written to:

\begin{verbatim}
~/.R/snippets/r.snippets
~/.R/snippets/c_cpp.snippets
\end{verbatim}
\end{frame}

\section{Tidyverse style}\label{tidyverse-style}

\begin{frame}{\href{https://style.tidyverse.org/index.html}{Tidyverse}}
\phantomsection\label{tidyverse}
Good coding style is like correct punctuation: you can manage without
it, butitsuremakesthingseasiertoread.

It was derived from \textbf{Google's original R Style Guide} - but
Google's current guide is derived from the tidyverse style guide.

\begin{itemize}
\item
  Files and Syntax
\item
  Functions
\item
  Pipes
\item
  ggplot2
\item
  packages
\end{itemize}
\end{frame}

\section{The Addins}\label{the-addins}

\begin{frame}{The Addins}
Addins are supper powerful tools to make your life easier.

Addins provide a mechanism for executing R functions interactively from
within the RStudio IDE---either through keyboard shortcuts, or through
the \textbf{Addins} menu.
\end{frame}

\begin{frame}[fragile]{Some Addins}
\phantomsection\label{some-addins}
\begin{enumerate}
\item
  \texttt{styler} allows you to interactively restyle selected text,
  files, or entire projects. It includes an RStudio add-in, the easiest
  way to re-style existing code.
\item
  \texttt{openai} for automatic spelling, code commenting and code
  creation. For automatic spelling, code commenting and code creation
\item
  \texttt{colourpicker} Especially for
  \texttt{colourpicker::colourWidget()}
\item
  \texttt{ymlthis} makes it easy to write YAML front matter for R
  Markdown and related documents.
\item
  \texttt{ViewPipeSteps} allows to print or view the output of your pipe
  chain after each step.
\end{enumerate}
\end{frame}

\section{QMD and Questions}\label{qmd-and-questions}

\begin{frame}{QMD and Questions}
Quarto is a multi-language, next-generation version of R Markdown

\begin{itemize}
\item
  Removes the bookdown pkg as a dependency in my scientific writing, it
  has simpler and universal cross-referencing
\item
  Multi-language and multi-engine: Python, R, and JavaScript via
  integration with Jupyter, Knitr, and Observable.
\item
  Global chunk options set in YAML, rather than a separate setup chunk
\item
  Easier to show verbatim chunks, good for writing tutorials and guides
\end{itemize}

\ldots{}
\end{frame}

\end{document}
